\title{Bài Tập SS004.L1.6}


\documentclass[a4paper,12pt]{article}

\usepackage{array}
\usepackage[T5]{fontenc}
\usepackage[utf8]{inputenc}
\usepackage[vietnamese,english]{babel}
\usepackage{amsmath}
\usepackage{amssymb}
\usepackage{amsthm}
\usepackage{graphicx}
\usepackage[colorinlistoftodos]{todonotes}
\usepackage{listings}
\usepackage{enumerate}
\usepackage[left=2cm,right=2cm,top=2cm,bottom=2cm]{geometry}
\usepackage{color}
\usepackage{hyperref}
\usepackage{longtable}
\usepackage[sort&compress]{natbib}
\hypersetup{
    colorlinks=true,
    linkcolor=blue,
    filecolor=magenta,      
    urlcolor=cyan,
}


\setlength{\marginparwidth}{2cm}
\begin{document}

\begin{titlepage}

\newcommand{\HRule}{\rule{\linewidth}{0.5mm}} % Defines a new command for the horizontal lines, change thickness here

\center % Center everything on the page
 

\textsc{\Huge Đại Học Công Nghệ Thông Tin}\\[1.5cm] % Name of your university/college
\textsc{\Large Ngành Kĩ Thuật Phần Mềm}\\[0.5cm] % Major heading such as course name
\textsc{\Large Môn Học: Kĩ năng nghề nghiệp }\\[0.5cm] % Minor heading such as course title

\HRule \\[0.4cm]
{ \huge \bfseries Bài Tập SS004.L1.6}\\[0.4cm] % Title of your document
\HRule \\[1.5cm]
 
\begin{minipage}{0.4\textwidth}
\begin{flushleft} \Large
\emph{Học viên:}\\
Võ Tấn Việt -- 19522519 \\
Võ Thành Phát -- 19522003 \\
Hồ Hoàng Phương -- 19522059 \\
Lớp: SS004.L16.CLC
\end{flushleft}
\end{minipage}
~
\begin{minipage}{0.4\textwidth}
\begin{flushright} \Large
\emph{Giảng viên:} \\
ThS. Nguyễn Văn Toàn % Supervisor's Name
\end{flushright}
\end{minipage}\\[2cm]


\includegraphics{49005358_1207866126005360_2030292411606892544_n.png}\\[1cm] % Include a department/university logo - this will require the graphicx package
 

\vfill % Fill the rest of the page with whitespace

\end{titlepage}




\Large

\section{Link trello và github của đồ án con rắn}

\begin{itemize}
     \item Trello: https://trello.com/b/6CUDzjQl/đồ-án-con-rắn
     
      \item Github: https://github.com/19522519/Baby-Snake 
\end{itemize} 

\section{Giới thiệu về chương trình/Game : giới thiệu sơ lược về các chức năng chính của trò chơi mà các SV viết, các class/struct chính, cấu trúc chính của chương trình}

\subsection{Mô tả game}
Trong game người chơi sẽ điều khiển con rắn ăn mồi là các dấu * . Mỗi khi ăn trúng mồi con rắn sẽ dài ra bằng một viên mồi mà nó vừa ăn được. Game chỉ thua khi người chơi tự để rắn ăn trúng thân mình hoặc va chạm vào các khung tường.

\subsection{Các hàm trong game}
\begin{itemize}
    \item Tạo khung game
    \item Tạo rắn và vẽ lên màn hình console
    \item Tạo cách di chuyển cho con rắn 
    \item Tạo con mồi một cách random
    \item Tạo các logic hợp lý ví dụ: khi rắn cắn vào người thì thua
    \item Tạo chế độ chơi: càng cao thì rắn càng đi nhanh
    \item Tạo menu chọn chơi và dừng khi người dùng ấn phím đặc biệt
\end{itemize}


\section{ Các điểm mà nhóm SV tâm đắc khi áp dụng các kỹ năng được biết trong việc
xây dựng trò chơi này.}
Tư duy lập trình được nâng cao khi lập trình game.
Thiết kế được độ khó cho trò chơi và cách di chuyển logic khi  ăn mồi, khi va  vào tường hay tự cắn bản thân của con rắn trong game.
Giao diện game đơn giản dễ tiếp cận.
Kỹ năng giao tiếp và làm việc nhóm tăng cao.
Sản phẩm nhóm đầu tiên của môn học.
  
  
\section{Hợp đồng nhóm}
    \subsection{TÊN NHÓM: Baby Snake.}

    \subsection{THÀNH VIÊN TRONG NHÓM}
    
    \centerline{\begin{tabular} { | c | c | c | }
    \hline
    Số thứ tự & Họ và tên & Mã số sinh viên \\
    \hline
    1  & Võ Tấn Việt  & 19522519  \\
    \hline
    2  & Hồ Hoàng Phương  & 19522059  \\
    \hline
    3  & Võ Thành Phát  & 19522003  \\
    \hline
    \end{tabular}}
    
    \subsection {MỤC ĐÍCH THÀNH LẬP NHÓM}
    
    \begin{itemize}
        \item Nâng cao kỹ năng làm việc nhóm cũng như các kỹ năng mềm khác.
        \item Cùng nhau hoàn thành bài đồ án cuối kì của bộ môn Kĩ năng nghề nghiệp.
        \item Hiểu rõ về cách làm thế nào để làm 1 game con rắn cũng như là hành trang cho con đường viết code phần mềm sau này.
        \item Hiểu được, định hướng được tương lai của bản thân sau khi ra trường thông qua môn học.
        \item Tạo sự đoàn kết, giúp đỡ nhau giữa các thành viên.
        \item Làm quen, biết cách ứng dụng phần mềm Latex.
        \item Làm quen các kĩ thuật quản lý dự án, quản lý tiến độ của các thành viên trong nhóm thông qua github và trello.
    \end{itemize}
    
    \subsection {VAI TRÒ CỦA TỪNG THÀNH VIÊN TRONG NHÓM}
    \begin{center}
    
    \begin{tabular} { | m{1.75cm} | m{1.75cm} | m{1.75cm} | m{1.75cm} | m{1.75cm} | m{1.75cm} | m{1.75cm} | }
    \hline
    Thành viên / Vai trò & Thái độ làm việc & Lãnh đạo nhóm và giữ tiến độ & Tìm kiếm thông tin và tổng hợp tài liệu & Thiết kế đồ họa & Có tính sáng tạo, nội dung phong phú & Trao đổi thông tin \\   
    \hline
    Võ Tấn Việt  & x & x & x &  & x & x \\
    \hline
    Võ Thành Phát  & x & x & x &  & x & x  \\
    \hline
    Hồ Hoàng Phương & x & x & x &  & x & x \\
    \hline
    \end{tabular}
    \end{center}
    
    \subsection {Nguyên tắc đánh giá}
    
    \begin{itemize}
        \item Đánh giá theo mức độ, chất lượng làm việc.
        \item Đánh giá chéo (Không thể tự đánh giá bản thân).
        \item Đã đọc qua nguyên tắc và chấp thuận những yêu cầu mà nhóm đã đặt ra.
        \item Bảng tiêu chí đánh giá thành viên trong nhóm:
    \end{itemize}
    
    \begin{center}
        \begin{tabular}{ | m{2.5cm} | m{2.95cm} | m{2.95cm} | m{2.95cm} | m{2.95cm} |}
            \hline
            Tiêu chí / Mức độ & Nổi bật & Tốt & Bình thường & Kém \\
            \hline
            Thái độ làm việc & Sẵn sàng nhận nhiệm vụ và hoàn thành vụ xuất sắc, tích cực giúp đỡ thành viên khác & Nhận nhiệm vụ và hoàn thành vụ ở mức tốt, có tham gia giúp đỡ các thành viên khác & Nhận nhiệm vụ và hoàn thành vụ của mình & Không chịu làm, không hoàn thành nhiệm vụ được giao hoặc hoàn thành qua loa sơ sài \\
            \hline
            Tìm kiếm thông tin và tổng hợp tài liệu & Tìm kiếm đầy đủ, phong phú các thông tin và tổng hợp 1 cách hiệu quả & Tìm kiếm đầy đủ, phong phú các thông tin & Tìm kiếm đầy đủ thông tin & Không tìm thông tin, báo cáo sơ sài \\
            \hline
            Thiết kế đồ họa & Trình bày gọn gàng, sạch sẽ, đa dạng, phong phú có tính sáng tạo cao & Trình bày gọn gàng, sạch sẽ, phong phú & Trình bày gọn gàng, sạch sẽ & Không thiết kế đồ họa hoặc trình bày cẩu thả \\
            \hline
            Trao đổi thông tin & Luôn chủ động liên lạc với nhóm, giúp đỡ các thành viên & Thường xuyên liên lạc với nhóm & Chỉ liên lạc với nhóm khi có ý kiến thắc mắc & Không liên lạc \\
            \hline
        \end{tabular}
    \end{center}
    
    \subsection {KẾT QUẢ ĐÁNH GIÁ CÁC THÀNH VIÊN TRONG NHÓM}
    \begin{center}
        \begin{tabular}{ | m{2.5cm} | m{2.25cm} | m{2.5cm} | m{2.25cm} | m{2.25cm} | m{2.5cm} |}
            \hline
            Họ và tên & Thái độ làm việc & Tìm kiếm thông tin và tổng hợp tài liệu & Thiết kế đồ họa & Trao đổi thông tin & Có tính sáng tạo, nội dung phong phú \\
            \hline
            Võ Tấn Việt & Nổi bật & Tốt & Tốt & Nỏi bật & Tốt  \\
            \hline
            Võ Thành Phát & Tốt & Nổi bật & Tốt & Nổi bật & Nổi bật \\
            \hline
            Hồ Hoàng Phương & Tốt & Nổi bật & Tốt & Tốt & Nổi bật \\
            \hline
        \end{tabular}
    \end{center}
    
\centerline{\textbf{CHỮ KÝ XÁC NHẬN ĐỒNG Ý VỚI KẾT QUẢ ĐÁNH GIÁ TRÊN}}

\begin{tabular}{m{5cm} m{5cm} m{5cm}}
    Võ Tấn Việt  &  Võ Thành Phát  &  Hồ Hoàng Phương \\
\end{tabular}

\end{document}